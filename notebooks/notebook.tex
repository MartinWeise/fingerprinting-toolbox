
% Default to the notebook output style

    


% Inherit from the specified cell style.




    
\documentclass[11pt]{article}

    
    
    \usepackage[T1]{fontenc}
    % Nicer default font (+ math font) than Computer Modern for most use cases
    \usepackage{mathpazo}

    % Basic figure setup, for now with no caption control since it's done
    % automatically by Pandoc (which extracts ![](path) syntax from Markdown).
    \usepackage{graphicx}
    % We will generate all images so they have a width \maxwidth. This means
    % that they will get their normal width if they fit onto the page, but
    % are scaled down if they would overflow the margins.
    \makeatletter
    \def\maxwidth{\ifdim\Gin@nat@width>\linewidth\linewidth
    \else\Gin@nat@width\fi}
    \makeatother
    \let\Oldincludegraphics\includegraphics
    % Set max figure width to be 80% of text width, for now hardcoded.
    \renewcommand{\includegraphics}[1]{\Oldincludegraphics[width=.8\maxwidth]{#1}}
    % Ensure that by default, figures have no caption (until we provide a
    % proper Figure object with a Caption API and a way to capture that
    % in the conversion process - todo).
    \usepackage{caption}
    \DeclareCaptionLabelFormat{nolabel}{}
    \captionsetup{labelformat=nolabel}

    \usepackage{adjustbox} % Used to constrain images to a maximum size 
    \usepackage{xcolor} % Allow colors to be defined
    \usepackage{enumerate} % Needed for markdown enumerations to work
    \usepackage{geometry} % Used to adjust the document margins
    \usepackage{amsmath} % Equations
    \usepackage{amssymb} % Equations
    \usepackage{textcomp} % defines textquotesingle
    % Hack from http://tex.stackexchange.com/a/47451/13684:
    \AtBeginDocument{%
        \def\PYZsq{\textquotesingle}% Upright quotes in Pygmentized code
    }
    \usepackage{upquote} % Upright quotes for verbatim code
    \usepackage{eurosym} % defines \euro
    \usepackage[mathletters]{ucs} % Extended unicode (utf-8) support
    \usepackage[utf8x]{inputenc} % Allow utf-8 characters in the tex document
    \usepackage{fancyvrb} % verbatim replacement that allows latex
    \usepackage{grffile} % extends the file name processing of package graphics 
                         % to support a larger range 
    % The hyperref package gives us a pdf with properly built
    % internal navigation ('pdf bookmarks' for the table of contents,
    % internal cross-reference links, web links for URLs, etc.)
    \usepackage{hyperref}
    \usepackage{longtable} % longtable support required by pandoc >1.10
    \usepackage{booktabs}  % table support for pandoc > 1.12.2
    \usepackage[inline]{enumitem} % IRkernel/repr support (it uses the enumerate* environment)
    \usepackage[normalem]{ulem} % ulem is needed to support strikethroughs (\sout)
                                % normalem makes italics be italics, not underlines
    

    
    
    % Colors for the hyperref package
    \definecolor{urlcolor}{rgb}{0,.145,.698}
    \definecolor{linkcolor}{rgb}{.71,0.21,0.01}
    \definecolor{citecolor}{rgb}{.12,.54,.11}

    % ANSI colors
    \definecolor{ansi-black}{HTML}{3E424D}
    \definecolor{ansi-black-intense}{HTML}{282C36}
    \definecolor{ansi-red}{HTML}{E75C58}
    \definecolor{ansi-red-intense}{HTML}{B22B31}
    \definecolor{ansi-green}{HTML}{00A250}
    \definecolor{ansi-green-intense}{HTML}{007427}
    \definecolor{ansi-yellow}{HTML}{DDB62B}
    \definecolor{ansi-yellow-intense}{HTML}{B27D12}
    \definecolor{ansi-blue}{HTML}{208FFB}
    \definecolor{ansi-blue-intense}{HTML}{0065CA}
    \definecolor{ansi-magenta}{HTML}{D160C4}
    \definecolor{ansi-magenta-intense}{HTML}{A03196}
    \definecolor{ansi-cyan}{HTML}{60C6C8}
    \definecolor{ansi-cyan-intense}{HTML}{258F8F}
    \definecolor{ansi-white}{HTML}{C5C1B4}
    \definecolor{ansi-white-intense}{HTML}{A1A6B2}

    % commands and environments needed by pandoc snippets
    % extracted from the output of `pandoc -s`
    \providecommand{\tightlist}{%
      \setlength{\itemsep}{0pt}\setlength{\parskip}{0pt}}
    \DefineVerbatimEnvironment{Highlighting}{Verbatim}{commandchars=\\\{\}}
    % Add ',fontsize=\small' for more characters per line
    \newenvironment{Shaded}{}{}
    \newcommand{\KeywordTok}[1]{\textcolor[rgb]{0.00,0.44,0.13}{\textbf{{#1}}}}
    \newcommand{\DataTypeTok}[1]{\textcolor[rgb]{0.56,0.13,0.00}{{#1}}}
    \newcommand{\DecValTok}[1]{\textcolor[rgb]{0.25,0.63,0.44}{{#1}}}
    \newcommand{\BaseNTok}[1]{\textcolor[rgb]{0.25,0.63,0.44}{{#1}}}
    \newcommand{\FloatTok}[1]{\textcolor[rgb]{0.25,0.63,0.44}{{#1}}}
    \newcommand{\CharTok}[1]{\textcolor[rgb]{0.25,0.44,0.63}{{#1}}}
    \newcommand{\StringTok}[1]{\textcolor[rgb]{0.25,0.44,0.63}{{#1}}}
    \newcommand{\CommentTok}[1]{\textcolor[rgb]{0.38,0.63,0.69}{\textit{{#1}}}}
    \newcommand{\OtherTok}[1]{\textcolor[rgb]{0.00,0.44,0.13}{{#1}}}
    \newcommand{\AlertTok}[1]{\textcolor[rgb]{1.00,0.00,0.00}{\textbf{{#1}}}}
    \newcommand{\FunctionTok}[1]{\textcolor[rgb]{0.02,0.16,0.49}{{#1}}}
    \newcommand{\RegionMarkerTok}[1]{{#1}}
    \newcommand{\ErrorTok}[1]{\textcolor[rgb]{1.00,0.00,0.00}{\textbf{{#1}}}}
    \newcommand{\NormalTok}[1]{{#1}}
    
    % Additional commands for more recent versions of Pandoc
    \newcommand{\ConstantTok}[1]{\textcolor[rgb]{0.53,0.00,0.00}{{#1}}}
    \newcommand{\SpecialCharTok}[1]{\textcolor[rgb]{0.25,0.44,0.63}{{#1}}}
    \newcommand{\VerbatimStringTok}[1]{\textcolor[rgb]{0.25,0.44,0.63}{{#1}}}
    \newcommand{\SpecialStringTok}[1]{\textcolor[rgb]{0.73,0.40,0.53}{{#1}}}
    \newcommand{\ImportTok}[1]{{#1}}
    \newcommand{\DocumentationTok}[1]{\textcolor[rgb]{0.73,0.13,0.13}{\textit{{#1}}}}
    \newcommand{\AnnotationTok}[1]{\textcolor[rgb]{0.38,0.63,0.69}{\textbf{\textit{{#1}}}}}
    \newcommand{\CommentVarTok}[1]{\textcolor[rgb]{0.38,0.63,0.69}{\textbf{\textit{{#1}}}}}
    \newcommand{\VariableTok}[1]{\textcolor[rgb]{0.10,0.09,0.49}{{#1}}}
    \newcommand{\ControlFlowTok}[1]{\textcolor[rgb]{0.00,0.44,0.13}{\textbf{{#1}}}}
    \newcommand{\OperatorTok}[1]{\textcolor[rgb]{0.40,0.40,0.40}{{#1}}}
    \newcommand{\BuiltInTok}[1]{{#1}}
    \newcommand{\ExtensionTok}[1]{{#1}}
    \newcommand{\PreprocessorTok}[1]{\textcolor[rgb]{0.74,0.48,0.00}{{#1}}}
    \newcommand{\AttributeTok}[1]{\textcolor[rgb]{0.49,0.56,0.16}{{#1}}}
    \newcommand{\InformationTok}[1]{\textcolor[rgb]{0.38,0.63,0.69}{\textbf{\textit{{#1}}}}}
    \newcommand{\WarningTok}[1]{\textcolor[rgb]{0.38,0.63,0.69}{\textbf{\textit{{#1}}}}}
    
    
    % Define a nice break command that doesn't care if a line doesn't already
    % exist.
    \def\br{\hspace*{\fill} \\* }
    % Math Jax compatability definitions
    \def\gt{>}
    \def\lt{<}
    % Document parameters
    \title{analysis}
    
    
    

    % Pygments definitions
    
\makeatletter
\def\PY@reset{\let\PY@it=\relax \let\PY@bf=\relax%
    \let\PY@ul=\relax \let\PY@tc=\relax%
    \let\PY@bc=\relax \let\PY@ff=\relax}
\def\PY@tok#1{\csname PY@tok@#1\endcsname}
\def\PY@toks#1+{\ifx\relax#1\empty\else%
    \PY@tok{#1}\expandafter\PY@toks\fi}
\def\PY@do#1{\PY@bc{\PY@tc{\PY@ul{%
    \PY@it{\PY@bf{\PY@ff{#1}}}}}}}
\def\PY#1#2{\PY@reset\PY@toks#1+\relax+\PY@do{#2}}

\expandafter\def\csname PY@tok@w\endcsname{\def\PY@tc##1{\textcolor[rgb]{0.73,0.73,0.73}{##1}}}
\expandafter\def\csname PY@tok@c\endcsname{\let\PY@it=\textit\def\PY@tc##1{\textcolor[rgb]{0.25,0.50,0.50}{##1}}}
\expandafter\def\csname PY@tok@cp\endcsname{\def\PY@tc##1{\textcolor[rgb]{0.74,0.48,0.00}{##1}}}
\expandafter\def\csname PY@tok@k\endcsname{\let\PY@bf=\textbf\def\PY@tc##1{\textcolor[rgb]{0.00,0.50,0.00}{##1}}}
\expandafter\def\csname PY@tok@kp\endcsname{\def\PY@tc##1{\textcolor[rgb]{0.00,0.50,0.00}{##1}}}
\expandafter\def\csname PY@tok@kt\endcsname{\def\PY@tc##1{\textcolor[rgb]{0.69,0.00,0.25}{##1}}}
\expandafter\def\csname PY@tok@o\endcsname{\def\PY@tc##1{\textcolor[rgb]{0.40,0.40,0.40}{##1}}}
\expandafter\def\csname PY@tok@ow\endcsname{\let\PY@bf=\textbf\def\PY@tc##1{\textcolor[rgb]{0.67,0.13,1.00}{##1}}}
\expandafter\def\csname PY@tok@nb\endcsname{\def\PY@tc##1{\textcolor[rgb]{0.00,0.50,0.00}{##1}}}
\expandafter\def\csname PY@tok@nf\endcsname{\def\PY@tc##1{\textcolor[rgb]{0.00,0.00,1.00}{##1}}}
\expandafter\def\csname PY@tok@nc\endcsname{\let\PY@bf=\textbf\def\PY@tc##1{\textcolor[rgb]{0.00,0.00,1.00}{##1}}}
\expandafter\def\csname PY@tok@nn\endcsname{\let\PY@bf=\textbf\def\PY@tc##1{\textcolor[rgb]{0.00,0.00,1.00}{##1}}}
\expandafter\def\csname PY@tok@ne\endcsname{\let\PY@bf=\textbf\def\PY@tc##1{\textcolor[rgb]{0.82,0.25,0.23}{##1}}}
\expandafter\def\csname PY@tok@nv\endcsname{\def\PY@tc##1{\textcolor[rgb]{0.10,0.09,0.49}{##1}}}
\expandafter\def\csname PY@tok@no\endcsname{\def\PY@tc##1{\textcolor[rgb]{0.53,0.00,0.00}{##1}}}
\expandafter\def\csname PY@tok@nl\endcsname{\def\PY@tc##1{\textcolor[rgb]{0.63,0.63,0.00}{##1}}}
\expandafter\def\csname PY@tok@ni\endcsname{\let\PY@bf=\textbf\def\PY@tc##1{\textcolor[rgb]{0.60,0.60,0.60}{##1}}}
\expandafter\def\csname PY@tok@na\endcsname{\def\PY@tc##1{\textcolor[rgb]{0.49,0.56,0.16}{##1}}}
\expandafter\def\csname PY@tok@nt\endcsname{\let\PY@bf=\textbf\def\PY@tc##1{\textcolor[rgb]{0.00,0.50,0.00}{##1}}}
\expandafter\def\csname PY@tok@nd\endcsname{\def\PY@tc##1{\textcolor[rgb]{0.67,0.13,1.00}{##1}}}
\expandafter\def\csname PY@tok@s\endcsname{\def\PY@tc##1{\textcolor[rgb]{0.73,0.13,0.13}{##1}}}
\expandafter\def\csname PY@tok@sd\endcsname{\let\PY@it=\textit\def\PY@tc##1{\textcolor[rgb]{0.73,0.13,0.13}{##1}}}
\expandafter\def\csname PY@tok@si\endcsname{\let\PY@bf=\textbf\def\PY@tc##1{\textcolor[rgb]{0.73,0.40,0.53}{##1}}}
\expandafter\def\csname PY@tok@se\endcsname{\let\PY@bf=\textbf\def\PY@tc##1{\textcolor[rgb]{0.73,0.40,0.13}{##1}}}
\expandafter\def\csname PY@tok@sr\endcsname{\def\PY@tc##1{\textcolor[rgb]{0.73,0.40,0.53}{##1}}}
\expandafter\def\csname PY@tok@ss\endcsname{\def\PY@tc##1{\textcolor[rgb]{0.10,0.09,0.49}{##1}}}
\expandafter\def\csname PY@tok@sx\endcsname{\def\PY@tc##1{\textcolor[rgb]{0.00,0.50,0.00}{##1}}}
\expandafter\def\csname PY@tok@m\endcsname{\def\PY@tc##1{\textcolor[rgb]{0.40,0.40,0.40}{##1}}}
\expandafter\def\csname PY@tok@gh\endcsname{\let\PY@bf=\textbf\def\PY@tc##1{\textcolor[rgb]{0.00,0.00,0.50}{##1}}}
\expandafter\def\csname PY@tok@gu\endcsname{\let\PY@bf=\textbf\def\PY@tc##1{\textcolor[rgb]{0.50,0.00,0.50}{##1}}}
\expandafter\def\csname PY@tok@gd\endcsname{\def\PY@tc##1{\textcolor[rgb]{0.63,0.00,0.00}{##1}}}
\expandafter\def\csname PY@tok@gi\endcsname{\def\PY@tc##1{\textcolor[rgb]{0.00,0.63,0.00}{##1}}}
\expandafter\def\csname PY@tok@gr\endcsname{\def\PY@tc##1{\textcolor[rgb]{1.00,0.00,0.00}{##1}}}
\expandafter\def\csname PY@tok@ge\endcsname{\let\PY@it=\textit}
\expandafter\def\csname PY@tok@gs\endcsname{\let\PY@bf=\textbf}
\expandafter\def\csname PY@tok@gp\endcsname{\let\PY@bf=\textbf\def\PY@tc##1{\textcolor[rgb]{0.00,0.00,0.50}{##1}}}
\expandafter\def\csname PY@tok@go\endcsname{\def\PY@tc##1{\textcolor[rgb]{0.53,0.53,0.53}{##1}}}
\expandafter\def\csname PY@tok@gt\endcsname{\def\PY@tc##1{\textcolor[rgb]{0.00,0.27,0.87}{##1}}}
\expandafter\def\csname PY@tok@err\endcsname{\def\PY@bc##1{\setlength{\fboxsep}{0pt}\fcolorbox[rgb]{1.00,0.00,0.00}{1,1,1}{\strut ##1}}}
\expandafter\def\csname PY@tok@kc\endcsname{\let\PY@bf=\textbf\def\PY@tc##1{\textcolor[rgb]{0.00,0.50,0.00}{##1}}}
\expandafter\def\csname PY@tok@kd\endcsname{\let\PY@bf=\textbf\def\PY@tc##1{\textcolor[rgb]{0.00,0.50,0.00}{##1}}}
\expandafter\def\csname PY@tok@kn\endcsname{\let\PY@bf=\textbf\def\PY@tc##1{\textcolor[rgb]{0.00,0.50,0.00}{##1}}}
\expandafter\def\csname PY@tok@kr\endcsname{\let\PY@bf=\textbf\def\PY@tc##1{\textcolor[rgb]{0.00,0.50,0.00}{##1}}}
\expandafter\def\csname PY@tok@bp\endcsname{\def\PY@tc##1{\textcolor[rgb]{0.00,0.50,0.00}{##1}}}
\expandafter\def\csname PY@tok@fm\endcsname{\def\PY@tc##1{\textcolor[rgb]{0.00,0.00,1.00}{##1}}}
\expandafter\def\csname PY@tok@vc\endcsname{\def\PY@tc##1{\textcolor[rgb]{0.10,0.09,0.49}{##1}}}
\expandafter\def\csname PY@tok@vg\endcsname{\def\PY@tc##1{\textcolor[rgb]{0.10,0.09,0.49}{##1}}}
\expandafter\def\csname PY@tok@vi\endcsname{\def\PY@tc##1{\textcolor[rgb]{0.10,0.09,0.49}{##1}}}
\expandafter\def\csname PY@tok@vm\endcsname{\def\PY@tc##1{\textcolor[rgb]{0.10,0.09,0.49}{##1}}}
\expandafter\def\csname PY@tok@sa\endcsname{\def\PY@tc##1{\textcolor[rgb]{0.73,0.13,0.13}{##1}}}
\expandafter\def\csname PY@tok@sb\endcsname{\def\PY@tc##1{\textcolor[rgb]{0.73,0.13,0.13}{##1}}}
\expandafter\def\csname PY@tok@sc\endcsname{\def\PY@tc##1{\textcolor[rgb]{0.73,0.13,0.13}{##1}}}
\expandafter\def\csname PY@tok@dl\endcsname{\def\PY@tc##1{\textcolor[rgb]{0.73,0.13,0.13}{##1}}}
\expandafter\def\csname PY@tok@s2\endcsname{\def\PY@tc##1{\textcolor[rgb]{0.73,0.13,0.13}{##1}}}
\expandafter\def\csname PY@tok@sh\endcsname{\def\PY@tc##1{\textcolor[rgb]{0.73,0.13,0.13}{##1}}}
\expandafter\def\csname PY@tok@s1\endcsname{\def\PY@tc##1{\textcolor[rgb]{0.73,0.13,0.13}{##1}}}
\expandafter\def\csname PY@tok@mb\endcsname{\def\PY@tc##1{\textcolor[rgb]{0.40,0.40,0.40}{##1}}}
\expandafter\def\csname PY@tok@mf\endcsname{\def\PY@tc##1{\textcolor[rgb]{0.40,0.40,0.40}{##1}}}
\expandafter\def\csname PY@tok@mh\endcsname{\def\PY@tc##1{\textcolor[rgb]{0.40,0.40,0.40}{##1}}}
\expandafter\def\csname PY@tok@mi\endcsname{\def\PY@tc##1{\textcolor[rgb]{0.40,0.40,0.40}{##1}}}
\expandafter\def\csname PY@tok@il\endcsname{\def\PY@tc##1{\textcolor[rgb]{0.40,0.40,0.40}{##1}}}
\expandafter\def\csname PY@tok@mo\endcsname{\def\PY@tc##1{\textcolor[rgb]{0.40,0.40,0.40}{##1}}}
\expandafter\def\csname PY@tok@ch\endcsname{\let\PY@it=\textit\def\PY@tc##1{\textcolor[rgb]{0.25,0.50,0.50}{##1}}}
\expandafter\def\csname PY@tok@cm\endcsname{\let\PY@it=\textit\def\PY@tc##1{\textcolor[rgb]{0.25,0.50,0.50}{##1}}}
\expandafter\def\csname PY@tok@cpf\endcsname{\let\PY@it=\textit\def\PY@tc##1{\textcolor[rgb]{0.25,0.50,0.50}{##1}}}
\expandafter\def\csname PY@tok@c1\endcsname{\let\PY@it=\textit\def\PY@tc##1{\textcolor[rgb]{0.25,0.50,0.50}{##1}}}
\expandafter\def\csname PY@tok@cs\endcsname{\let\PY@it=\textit\def\PY@tc##1{\textcolor[rgb]{0.25,0.50,0.50}{##1}}}

\def\PYZbs{\char`\\}
\def\PYZus{\char`\_}
\def\PYZob{\char`\{}
\def\PYZcb{\char`\}}
\def\PYZca{\char`\^}
\def\PYZam{\char`\&}
\def\PYZlt{\char`\<}
\def\PYZgt{\char`\>}
\def\PYZsh{\char`\#}
\def\PYZpc{\char`\%}
\def\PYZdl{\char`\$}
\def\PYZhy{\char`\-}
\def\PYZsq{\char`\'}
\def\PYZdq{\char`\"}
\def\PYZti{\char`\~}
% for compatibility with earlier versions
\def\PYZat{@}
\def\PYZlb{[}
\def\PYZrb{]}
\makeatother


    % Exact colors from NB
    \definecolor{incolor}{rgb}{0.0, 0.0, 0.5}
    \definecolor{outcolor}{rgb}{0.545, 0.0, 0.0}



    
    % Prevent overflowing lines due to hard-to-break entities
    \sloppy 
    % Setup hyperref package
    \hypersetup{
      breaklinks=true,  % so long urls are correctly broken across lines
      colorlinks=true,
      urlcolor=urlcolor,
      linkcolor=linkcolor,
      citecolor=citecolor,
      }
    % Slightly bigger margins than the latex defaults
    
    \geometry{verbose,tmargin=1in,bmargin=1in,lmargin=1in,rmargin=1in}
    
    

    \begin{document}
    
    
    \maketitle
    
    

    
    \hypertarget{fingerprinting-categorical-data}{%
\section{Fingerprinting Categorical
Data}\label{fingerprinting-categorical-data}}

\hypertarget{nearest-neighbours-approach}{%
\subsection{Nearest Neighbours
Approach}\label{nearest-neighbours-approach}}

    In this notebook we analyse the behaviour of the fingerprinting
technique for categorical data. The approach examines the neighbourhood
of the observed item to decide how to mark it. The technique works for
numerical type of data as well - in this setting the AK scheme is used.
In the settings where the dataset contains a mix of data types, the
techniques are chosen dynamically, based on the value chosen for
marking.

    \begin{Verbatim}[commandchars=\\\{\}]
{\color{incolor}In [{\color{incolor}1}]:} \PY{c+c1}{\PYZsh{} This is important for module import }
        \PY{k+kn}{import} \PY{n+nn}{sys}\PY{o}{,} \PY{n+nn}{os}
        \PY{k}{if} \PY{l+s+s1}{\PYZsq{}}\PY{l+s+s1}{C:/Users/tsarcevic/PycharmProjects/fingerprinting\PYZhy{}toolbox/}\PY{l+s+s1}{\PYZsq{}} \PY{o+ow}{not} \PY{o+ow}{in} \PY{n}{sys}\PY{o}{.}\PY{n}{path}\PY{p}{:}
            \PY{n}{sys}\PY{o}{.}\PY{n}{path}\PY{o}{.}\PY{n}{append}\PY{p}{(}\PY{l+s+s1}{\PYZsq{}}\PY{l+s+s1}{C:/Users/tsarcevic/PycharmProjects/fingerprinting\PYZhy{}toolbox/}\PY{l+s+s1}{\PYZsq{}}\PY{p}{)}
        \PY{n}{os}\PY{o}{.}\PY{n}{chdir}\PY{p}{(}\PY{l+s+s1}{\PYZsq{}}\PY{l+s+s1}{../}\PY{l+s+s1}{\PYZsq{}}\PY{p}{)}
\end{Verbatim}


    \begin{Verbatim}[commandchars=\\\{\}]
{\color{incolor}In [{\color{incolor}2}]:} \PY{k+kn}{import} \PY{n+nn}{warnings}
        \PY{n}{warnings}\PY{o}{.}\PY{n}{filterwarnings}\PY{p}{(}\PY{l+s+s2}{\PYZdq{}}\PY{l+s+s2}{ignore}\PY{l+s+s2}{\PYZdq{}}\PY{p}{)}
\end{Verbatim}


    \begin{Verbatim}[commandchars=\\\{\}]
{\color{incolor}In [{\color{incolor}37}]:} \PY{k+kn}{import} \PY{n+nn}{pandas} \PY{k}{as} \PY{n+nn}{pd}
         \PY{k+kn}{import} \PY{n+nn}{matplotlib}\PY{n+nn}{.}\PY{n+nn}{pyplot} \PY{k}{as} \PY{n+nn}{plt}
         \PY{k+kn}{import} \PY{n+nn}{seaborn} \PY{k}{as} \PY{n+nn}{sns}
\end{Verbatim}


    \begin{Verbatim}[commandchars=\\\{\}]
{\color{incolor}In [{\color{incolor}4}]:} \PY{k+kn}{from} \PY{n+nn}{schemes}\PY{n+nn}{.}\PY{n+nn}{categorical\PYZus{}neighbourhood}\PY{n+nn}{.}\PY{n+nn}{categorical\PYZus{}neighbourhood} \PY{k}{import} \PY{n}{CategoricalNeighbourhood}
        \PY{k+kn}{from} \PY{n+nn}{utils} \PY{k}{import} \PY{o}{*}
\end{Verbatim}


    Let us take a look at our data.

    \begin{Verbatim}[commandchars=\\\{\}]
{\color{incolor}In [{\color{incolor}5}]:} \PY{n}{dataset\PYZus{}name} \PY{o}{=} \PY{l+s+s2}{\PYZdq{}}\PY{l+s+s2}{german\PYZus{}credit\PYZus{}sample}\PY{l+s+s2}{\PYZdq{}}
        \PY{n}{dataset}\PY{p}{,} \PY{n}{primary\PYZus{}key} \PY{o}{=} \PY{n}{import\PYZus{}dataset}\PY{p}{(}\PY{n}{dataset\PYZus{}name}\PY{p}{)}
        \PY{n}{dataset}
\end{Verbatim}


    \begin{Verbatim}[commandchars=\\\{\}]
Dataset: datasets/german\_credit\_sample.csv

    \end{Verbatim}

\begin{Verbatim}[commandchars=\\\{\}]
{\color{outcolor}Out[{\color{outcolor}5}]:}     Id checking\_account  duration credit\_hist purpose  credit\_amount savings  \textbackslash{}
        0    0              A11         6         A34     A43           1169     A65   
        1    1              A12        48         A32     A43           5951     A61   
        2    2              A14        12         A34     A46           2096     A61   
        3    3              A11        42         A32     A42           7882     A61   
        4    4              A11        24         A33     A40           4870     A61   
        5    5              A14        36         A32     A46           9055     A65   
        6    6              A14        24         A32     A42           2835     A63   
        7    7              A12        36         A32     A41           6948     A61   
        8    8              A14        12         A32     A43           3059     A64   
        9    9              A12        30         A34     A40           5234     A61   
        10  10              A12        12         A32     A40           1295     A61   
        11  11              A11        48         A32     A49           4308     A61   
        12  12              A12        12         A32     A43           1567     A61   
        13  13              A11        24         A34     A40           1199     A61   
        14  14              A11        15         A32     A40           1403     A61   
        15  15              A11        24         A32     A43           1282     A62   
        16  16              A14        24         A34     A43           2424     A65   
        17  17              A11        30         A30     A49           8072     A65   
        18  18              A12        24         A32     A41          12579     A61   
        19  19              A14        24         A32     A43           3430     A63   
        20  20              A14         9         A34     A40           2134     A61   
        21  21              A11         6         A32     A43           2647     A63   
        22  22              A11        10         A34     A40           2241     A61   
        23  23              A12        12         A34     A41           1804     A62   
        24  24              A14        10         A34     A42           2069     A65   
        25  25              A11         6         A32     A42           1374     A61   
        26  26              A14         6         A30     A43            426     A61   
        27  27              A13        12         A31     A43            409     A64   
        28  28              A12         7         A32     A43           2415     A61   
        29  29              A11        60         A33     A49           6836     A61   
        
           employment\_since  installment\_rate sex\_status debtors  residence\_since  
        0               A75                 4        A93    A101                4  
        1               A73                 2        A92    A101                2  
        2               A74                 2        A93    A101                3  
        3               A74                 2        A93    A103                4  
        4               A73                 3        A93    A101                4  
        5               A73                 2        A93    A101                4  
        6               A75                 3        A93    A101                4  
        7               A73                 2        A93    A101                2  
        8               A74                 2        A91    A101                4  
        9               A71                 4        A94    A101                2  
        10              A72                 3        A92    A101                1  
        11              A72                 3        A92    A101                4  
        12              A73                 1        A92    A101                1  
        13              A75                 4        A93    A101                4  
        14              A73                 2        A92    A101                4  
        15              A73                 4        A92    A101                2  
        16              A75                 4        A93    A101                4  
        17              A72                 2        A93    A101                3  
        18              A75                 4        A92    A101                2  
        19              A75                 3        A93    A101                2  
        20              A73                 4        A93    A101                4  
        21              A73                 2        A93    A101                3  
        22              A72                 1        A93    A101                3  
        23              A72                 3        A93    A101                4  
        24              A73                 2        A94    A101                1  
        25              A73                 1        A93    A101                2  
        26              A75                 4        A94    A101                4  
        27              A73                 3        A92    A101                3  
        28              A73                 3        A93    A103                2  
        29              A75                 3        A93    A101                4  
\end{Verbatim}
            
    We embedd the fingerprint into our dataset. The fingerprint will be 32
bits long, and it is embedded in every third rown in the dataset on
average. The secret key is known only to the owner of the dataset and
without it is impossible to recreate the algoritm.

    \begin{Verbatim}[commandchars=\\\{\}]
{\color{incolor}In [{\color{incolor}6}]:} \PY{n}{scheme} \PY{o}{=} \PY{n}{CategoricalNeighbourhood}\PY{p}{(}\PY{n}{gamma}\PY{o}{=}\PY{l+m+mi}{3}\PY{p}{,} \PY{n}{xi}\PY{o}{=}\PY{l+m+mi}{2}\PY{p}{,} \PY{n}{fingerprint\PYZus{}bit\PYZus{}length}\PY{o}{=}\PY{l+m+mi}{32}\PY{p}{,} \PY{n}{number\PYZus{}of\PYZus{}buyers}\PY{o}{=}\PY{l+m+mi}{10}\PY{p}{,} \PY{n}{secret\PYZus{}key}\PY{o}{=}\PY{l+m+mi}{333}\PY{p}{,} \PY{n}{k}\PY{o}{=}\PY{l+m+mi}{5}\PY{p}{)}
        \PY{n}{scheme}\PY{o}{.}\PY{n}{insertion}\PY{p}{(}\PY{n}{dataset\PYZus{}name}\PY{o}{=}\PY{n}{dataset\PYZus{}name}\PY{p}{,} \PY{n}{buyer\PYZus{}id}\PY{o}{=}\PY{l+m+mi}{0}\PY{p}{)}
\end{Verbatim}


    \begin{Verbatim}[commandchars=\\\{\}]
Start the insertion algorithm of a scheme for fingerprinting categorical data (neighbourhood) {\ldots}
	gamma: 3
	xi: 2
Dataset: datasets/german\_credit\_sample.csv

Generated fingerprint for buyer 0: 01001110011110110000100110110000
Inserting the fingerprint{\ldots}

Training balltrees in: 0.02 sec.
Size of a neighbourhood: 10 instead of 5
	Neighbours: [13, 14, 29, 22, 10, 3, 9, 11, 25, 20]
Size of a neighbourhood: 8 instead of 5
	Neighbours: [0, 24, 2, 8, 26, 20, 5, 19]
Size of a neighbourhood: 12 instead of 5
	Neighbours: [29, 11, 0, 3, 13, 14, 15, 4, 21, 25, 26, 22]
Fingerprint inserted.
	fingerprinted dataset written to: schemes/categorical\_neighbourhood/fingerprinted\_datasets/german\_credit\_sample\_3\_2\_0.csv
Time: 0 sec.

    \end{Verbatim}

    \begin{Verbatim}[commandchars=\\\{\}]
c:\textbackslash{}users\textbackslash{}tsarcevic\textbackslash{}appdata\textbackslash{}local\textbackslash{}programs\textbackslash{}python\textbackslash{}python37-32\textbackslash{}lib\textbackslash{}site-packages\textbackslash{}sklearn\textbackslash{}preprocessing\textbackslash{}label.py:151: DeprecationWarning: The truth value of an empty array is ambiguous. Returning False, but in future this will result in an error. Use `array.size > 0` to check that an array is not empty.
  if diff:
c:\textbackslash{}users\textbackslash{}tsarcevic\textbackslash{}appdata\textbackslash{}local\textbackslash{}programs\textbackslash{}python\textbackslash{}python37-32\textbackslash{}lib\textbackslash{}site-packages\textbackslash{}sklearn\textbackslash{}preprocessing\textbackslash{}label.py:151: DeprecationWarning: The truth value of an empty array is ambiguous. Returning False, but in future this will result in an error. Use `array.size > 0` to check that an array is not empty.
  if diff:
c:\textbackslash{}users\textbackslash{}tsarcevic\textbackslash{}appdata\textbackslash{}local\textbackslash{}programs\textbackslash{}python\textbackslash{}python37-32\textbackslash{}lib\textbackslash{}site-packages\textbackslash{}sklearn\textbackslash{}preprocessing\textbackslash{}label.py:151: DeprecationWarning: The truth value of an empty array is ambiguous. Returning False, but in future this will result in an error. Use `array.size > 0` to check that an array is not empty.
  if diff:
c:\textbackslash{}users\textbackslash{}tsarcevic\textbackslash{}appdata\textbackslash{}local\textbackslash{}programs\textbackslash{}python\textbackslash{}python37-32\textbackslash{}lib\textbackslash{}site-packages\textbackslash{}sklearn\textbackslash{}preprocessing\textbackslash{}label.py:151: DeprecationWarning: The truth value of an empty array is ambiguous. Returning False, but in future this will result in an error. Use `array.size > 0` to check that an array is not empty.
  if diff:
c:\textbackslash{}users\textbackslash{}tsarcevic\textbackslash{}appdata\textbackslash{}local\textbackslash{}programs\textbackslash{}python\textbackslash{}python37-32\textbackslash{}lib\textbackslash{}site-packages\textbackslash{}sklearn\textbackslash{}preprocessing\textbackslash{}label.py:151: DeprecationWarning: The truth value of an empty array is ambiguous. Returning False, but in future this will result in an error. Use `array.size > 0` to check that an array is not empty.
  if diff:
c:\textbackslash{}users\textbackslash{}tsarcevic\textbackslash{}appdata\textbackslash{}local\textbackslash{}programs\textbackslash{}python\textbackslash{}python37-32\textbackslash{}lib\textbackslash{}site-packages\textbackslash{}sklearn\textbackslash{}preprocessing\textbackslash{}label.py:151: DeprecationWarning: The truth value of an empty array is ambiguous. Returning False, but in future this will result in an error. Use `array.size > 0` to check that an array is not empty.
  if diff:
c:\textbackslash{}users\textbackslash{}tsarcevic\textbackslash{}appdata\textbackslash{}local\textbackslash{}programs\textbackslash{}python\textbackslash{}python37-32\textbackslash{}lib\textbackslash{}site-packages\textbackslash{}sklearn\textbackslash{}preprocessing\textbackslash{}label.py:151: DeprecationWarning: The truth value of an empty array is ambiguous. Returning False, but in future this will result in an error. Use `array.size > 0` to check that an array is not empty.
  if diff:

    \end{Verbatim}

\begin{Verbatim}[commandchars=\\\{\}]
{\color{outcolor}Out[{\color{outcolor}6}]:} True
\end{Verbatim}
            
    The fingerprint is inserted in the dataset. We can now extract the
differences to see the effect of the fingerprint.

    \begin{Verbatim}[commandchars=\\\{\}]
{\color{incolor}In [{\color{incolor}7}]:} \PY{n}{fingerprinted\PYZus{}data}\PY{p}{,} \PY{n}{primary\PYZus{}key} \PY{o}{=} \PY{n}{import\PYZus{}fingerprinted\PYZus{}dataset}\PY{p}{(}\PY{l+s+s2}{\PYZdq{}}\PY{l+s+s2}{categorical\PYZus{}neighbourhood}\PY{l+s+s2}{\PYZdq{}}\PY{p}{,} \PY{n}{dataset\PYZus{}name}\PY{p}{,} \PY{p}{[}\PY{l+m+mi}{3}\PY{p}{,} \PY{l+m+mi}{2}\PY{p}{]}\PY{p}{,} \PY{l+m+mi}{0}\PY{p}{)}
\end{Verbatim}


    \begin{Verbatim}[commandchars=\\\{\}]
Dataset: schemes/categorical\_neighbourhood/fingerprinted\_datasets/german\_credit\_sample\_3\_2\_0.csv

    \end{Verbatim}

    \begin{Verbatim}[commandchars=\\\{\}]
{\color{incolor}In [{\color{incolor}8}]:} \PY{n}{diff} \PY{o}{=} \PY{n}{dataset}\PY{p}{[}\PY{o}{\PYZti{}}\PY{n}{dataset}\PY{o}{.}\PY{n}{isin}\PY{p}{(}\PY{n}{fingerprinted\PYZus{}data}\PY{p}{)}\PY{p}{]}\PY{o}{.}\PY{n}{dropna}\PY{p}{(}\PY{n}{how}\PY{o}{=}\PY{l+s+s2}{\PYZdq{}}\PY{l+s+s2}{all}\PY{l+s+s2}{\PYZdq{}}\PY{p}{)}\PY{o}{.}\PY{n}{index}
        
        \PY{n}{combo} \PY{o}{=} \PY{n}{pd}\PY{o}{.}\PY{n}{concat}\PY{p}{(}\PY{p}{[}\PY{n}{dataset}\PY{o}{.}\PY{n}{loc}\PY{p}{[}\PY{n}{diff}\PY{p}{]}\PY{p}{,} \PY{n}{fingerprinted\PYZus{}data}\PY{o}{.}\PY{n}{loc}\PY{p}{[}\PY{n}{diff}\PY{p}{]}\PY{p}{]}\PY{p}{)}
        \PY{n}{combo} \PY{o}{=} \PY{n}{combo}\PY{o}{.}\PY{n}{sort\PYZus{}index}\PY{p}{(}\PY{p}{)}
        \PY{n}{combo}
\end{Verbatim}


\begin{Verbatim}[commandchars=\\\{\}]
{\color{outcolor}Out[{\color{outcolor}8}]:}     Id checking\_account  duration credit\_hist purpose  credit\_amount savings  \textbackslash{}
        2    2              A14        12         A34     A46           2096     A61   
        2    2              A14        13         A34     A46           2096     A61   
        4    4              A11        24         A33     A40           4870     A61   
        4    4              A11        24         A33     A40           4870     A61   
        16  16              A14        24         A34     A43           2424     A65   
        16  16              A14        24         A34     A43           2424     A65   
        17  17              A11        30         A30     A49           8072     A65   
        17  17              A11        30         A30     A49           8072     A61   
        24  24              A14        10         A34     A42           2069     A65   
        24  24              A14        10         A34     A42           2069     A65   
        
           employment\_since  installment\_rate sex\_status debtors  residence\_since  
        2               A74                 2        A93    A101                3  
        2               A74                 2        A93    A101                3  
        4               A73                 3        A93    A101                4  
        4               A73                 3        A93    A103                4  
        16              A75                 4        A93    A101                4  
        16              A75                 4        A94    A101                4  
        17              A72                 2        A93    A101                3  
        17              A72                 2        A93    A101                3  
        24              A73                 2        A94    A101                1  
        24              A73                 0        A94    A101                1  
\end{Verbatim}
            
    There are 5 differences in total - 2 between numerical values, 3 between
categorical. Let us take for example the row with id=4 and examine how
the mark value was chosen (A101-\textgreater{}A103). For now our
algorithm works such that for neighbourhood search it considers all but
last three columns. The insertion algorithm looks at the values of row
id=4 and searches for 5 neighbours. In this case many of them have the
same distance, so k is increased, in this case to 10. We have our
neighbourhood: rows 13, 14, 29, 22, 10, 3, 9, 11, 25, 20. Next thing the
algorithm does is count the frequencies of our target values in the
neihgbourhood. Here we plot the frequencies:

    \begin{Verbatim}[commandchars=\\\{\}]
{\color{incolor}In [{\color{incolor}136}]:} \PY{n}{fig}\PY{p}{,} \PY{n}{ax} \PY{o}{=} \PY{n}{plt}\PY{o}{.}\PY{n}{subplots}\PY{p}{(}\PY{l+m+mi}{1}\PY{p}{,}\PY{l+m+mi}{2}\PY{p}{,} \PY{n}{figsize}\PY{o}{=}\PY{p}{(}\PY{l+m+mi}{8}\PY{p}{,}\PY{l+m+mi}{3}\PY{p}{)}\PY{p}{)}
          \PY{n}{counts} \PY{o}{=} \PY{n}{dataset}\PY{p}{[}\PY{l+s+s2}{\PYZdq{}}\PY{l+s+s2}{debtors}\PY{l+s+s2}{\PYZdq{}}\PY{p}{]}\PY{p}{[}\PY{p}{[}\PY{l+m+mi}{13}\PY{p}{,}\PY{l+m+mi}{14}\PY{p}{,}\PY{l+m+mi}{29}\PY{p}{,}\PY{l+m+mi}{22}\PY{p}{,}\PY{l+m+mi}{10}\PY{p}{,}\PY{l+m+mi}{3}\PY{p}{,}\PY{l+m+mi}{9}\PY{p}{,}\PY{l+m+mi}{11}\PY{p}{,}\PY{l+m+mi}{25}\PY{p}{,}\PY{l+m+mi}{20}\PY{p}{]}\PY{p}{]}\PY{o}{.}\PY{n}{value\PYZus{}counts}\PY{p}{(}\PY{p}{)}
          \PY{n}{sns}\PY{o}{.}\PY{n}{barplot}\PY{p}{(}\PY{n}{counts}\PY{o}{.}\PY{n}{index}\PY{p}{,} \PY{n}{counts}\PY{o}{.}\PY{n}{values}\PY{p}{,} \PY{n}{alpha}\PY{o}{=}\PY{o}{.}\PY{l+m+mi}{5}\PY{p}{,} \PY{n}{palette}\PY{o}{=}\PY{l+s+s2}{\PYZdq{}}\PY{l+s+s2}{Set1}\PY{l+s+s2}{\PYZdq{}}\PY{p}{,} \PY{n}{ax}\PY{o}{=}\PY{n}{ax}\PY{p}{[}\PY{l+m+mi}{0}\PY{p}{]}\PY{p}{)}
          
          \PY{n}{counts} \PY{o}{=} \PY{n}{counts}\PY{o}{.}\PY{n}{drop}\PY{p}{(}\PY{l+s+s2}{\PYZdq{}}\PY{l+s+s2}{A101}\PY{l+s+s2}{\PYZdq{}}\PY{p}{)}
          \PY{n}{sns}\PY{o}{.}\PY{n}{barplot}\PY{p}{(}\PY{n}{counts}\PY{o}{.}\PY{n}{index}\PY{p}{,} \PY{n}{counts}\PY{o}{.}\PY{n}{values}\PY{p}{,} \PY{n}{alpha}\PY{o}{=}\PY{o}{.}\PY{l+m+mi}{95}\PY{p}{,} \PY{n}{palette}\PY{o}{=}\PY{l+s+s2}{\PYZdq{}}\PY{l+s+s2}{Set1}\PY{l+s+s2}{\PYZdq{}}\PY{p}{,} \PY{n}{ax}\PY{o}{=}\PY{n}{ax}\PY{p}{[}\PY{l+m+mi}{1}\PY{p}{]}\PY{p}{)}
\end{Verbatim}


\begin{Verbatim}[commandchars=\\\{\}]
{\color{outcolor}Out[{\color{outcolor}136}]:} <matplotlib.axes.\_subplots.AxesSubplot at 0x23e286f0>
\end{Verbatim}
            
    \begin{center}
    \adjustimage{max size={0.9\linewidth}{0.9\paperheight}}{output_14_1.png}
    \end{center}
    { \hspace*{\fill} \\}
    
    The first plot represents the taget values ``deptors'' in the
neighbourhood. We see that value A101 occurs 9 times and the value A103
1 time. Since the value A101 is the original value, we force the change
as a mark and obtain only value A103 (the plot on the right). This value
is chosen as a mark since it's the only remaining value.

Let us see other marked categorical values. The next one is in the row
with id=16 and the change occurs for attribute sex\_status from value
A93 to A94. The process of marking here was the same as in the previous
example. The neighbourhood is calculated and frequencies of the target
value obtained:

    \begin{Verbatim}[commandchars=\\\{\}]
{\color{incolor}In [{\color{incolor}137}]:} \PY{n}{fig}\PY{p}{,} \PY{n}{ax} \PY{o}{=} \PY{n}{plt}\PY{o}{.}\PY{n}{subplots}\PY{p}{(}\PY{l+m+mi}{1}\PY{p}{,}\PY{l+m+mi}{2}\PY{p}{,} \PY{n}{figsize}\PY{o}{=}\PY{p}{(}\PY{l+m+mi}{8}\PY{p}{,}\PY{l+m+mi}{3}\PY{p}{)}\PY{p}{)}
          \PY{n}{counts} \PY{o}{=} \PY{n}{dataset}\PY{p}{[}\PY{l+s+s2}{\PYZdq{}}\PY{l+s+s2}{sex\PYZus{}status}\PY{l+s+s2}{\PYZdq{}}\PY{p}{]}\PY{p}{[}\PY{p}{[}\PY{l+m+mi}{0}\PY{p}{,} \PY{l+m+mi}{24}\PY{p}{,} \PY{l+m+mi}{2}\PY{p}{,} \PY{l+m+mi}{8}\PY{p}{,} \PY{l+m+mi}{26}\PY{p}{,} \PY{l+m+mi}{20}\PY{p}{,} \PY{l+m+mi}{5}\PY{p}{,} \PY{l+m+mi}{19}\PY{p}{]}\PY{p}{]}\PY{o}{.}\PY{n}{value\PYZus{}counts}\PY{p}{(}\PY{p}{)}
          \PY{n}{sns}\PY{o}{.}\PY{n}{barplot}\PY{p}{(}\PY{n}{counts}\PY{o}{.}\PY{n}{index}\PY{p}{,} \PY{n}{counts}\PY{o}{.}\PY{n}{values}\PY{p}{,} \PY{n}{alpha}\PY{o}{=}\PY{o}{.}\PY{l+m+mi}{5}\PY{p}{,} \PY{n}{palette}\PY{o}{=}\PY{l+s+s2}{\PYZdq{}}\PY{l+s+s2}{Set2}\PY{l+s+s2}{\PYZdq{}}\PY{p}{,} \PY{n}{ax}\PY{o}{=}\PY{n}{ax}\PY{p}{[}\PY{l+m+mi}{0}\PY{p}{]}\PY{p}{)}
          
          \PY{n}{counts} \PY{o}{=} \PY{n}{counts}\PY{o}{.}\PY{n}{drop}\PY{p}{(}\PY{l+s+s2}{\PYZdq{}}\PY{l+s+s2}{A93}\PY{l+s+s2}{\PYZdq{}}\PY{p}{)}
          \PY{n}{sns}\PY{o}{.}\PY{n}{barplot}\PY{p}{(}\PY{n}{counts}\PY{o}{.}\PY{n}{index}\PY{p}{,} \PY{n}{counts}\PY{o}{.}\PY{n}{values}\PY{p}{,} \PY{n}{alpha}\PY{o}{=}\PY{o}{.}\PY{l+m+mi}{95}\PY{p}{,} \PY{n}{palette}\PY{o}{=}\PY{l+s+s2}{\PYZdq{}}\PY{l+s+s2}{Set2}\PY{l+s+s2}{\PYZdq{}}\PY{p}{,} \PY{n}{ax}\PY{o}{=}\PY{n}{ax}\PY{p}{[}\PY{l+m+mi}{1}\PY{p}{]}\PY{p}{)}
\end{Verbatim}


\begin{Verbatim}[commandchars=\\\{\}]
{\color{outcolor}Out[{\color{outcolor}137}]:} <matplotlib.axes.\_subplots.AxesSubplot at 0x23e7ee90>
\end{Verbatim}
            
    \begin{center}
    \adjustimage{max size={0.9\linewidth}{0.9\paperheight}}{output_16_1.png}
    \end{center}
    { \hspace*{\fill} \\}
    
    The plot on the left shows the frequencies of values of attribute
``sex\_status'' in the neighbourhood: A93 occurs 5 times, A94 2 times
and A91 one time (neighbourhood size is increased again; now it's 8).
Since the original value is A93, we want to remove it from consideration
so we obtain the plot on the right. There are two values to choose from.
The algorithm weights it's random choice of mark by it's frequency,
therefore A94 has more chance to be chosen as a mark than A91 - indeed
it is chosen in this algorithm run.

The remaining example of marking categorical value follows the same
steps as shown above. We have a mark in a row 17, attribute savings (A65
-\textgreater{} A61). We obtain the frequencis from the neighbourhood:

    \begin{Verbatim}[commandchars=\\\{\}]
{\color{incolor}In [{\color{incolor}138}]:} \PY{n}{fig}\PY{p}{,} \PY{n}{ax} \PY{o}{=} \PY{n}{plt}\PY{o}{.}\PY{n}{subplots}\PY{p}{(}\PY{l+m+mi}{1}\PY{p}{,}\PY{l+m+mi}{2}\PY{p}{,} \PY{n}{figsize}\PY{o}{=}\PY{p}{(}\PY{l+m+mi}{8}\PY{p}{,}\PY{l+m+mi}{3}\PY{p}{)}\PY{p}{)}
          \PY{n}{counts} \PY{o}{=} \PY{n}{dataset}\PY{p}{[}\PY{l+s+s2}{\PYZdq{}}\PY{l+s+s2}{savings}\PY{l+s+s2}{\PYZdq{}}\PY{p}{]}\PY{p}{[}\PY{p}{[}\PY{l+m+mi}{29}\PY{p}{,} \PY{l+m+mi}{11}\PY{p}{,} \PY{l+m+mi}{0}\PY{p}{,} \PY{l+m+mi}{3}\PY{p}{,} \PY{l+m+mi}{13}\PY{p}{,} \PY{l+m+mi}{14}\PY{p}{,} \PY{l+m+mi}{15}\PY{p}{,} \PY{l+m+mi}{4}\PY{p}{,} \PY{l+m+mi}{21}\PY{p}{,} \PY{l+m+mi}{25}\PY{p}{,} \PY{l+m+mi}{26}\PY{p}{,} \PY{l+m+mi}{22}\PY{p}{]}\PY{p}{]}\PY{o}{.}\PY{n}{value\PYZus{}counts}\PY{p}{(}\PY{p}{)}
          \PY{n}{sns}\PY{o}{.}\PY{n}{barplot}\PY{p}{(}\PY{n}{counts}\PY{o}{.}\PY{n}{index}\PY{p}{,}  \PY{n}{counts}\PY{o}{.}\PY{n}{values}\PY{p}{,} \PY{n}{alpha}\PY{o}{=}\PY{o}{.}\PY{l+m+mi}{5}\PY{p}{,} \PY{n}{palette}\PY{o}{=}\PY{l+s+s2}{\PYZdq{}}\PY{l+s+s2}{Set3}\PY{l+s+s2}{\PYZdq{}}\PY{p}{,} \PY{n}{ax}\PY{o}{=}\PY{n}{ax}\PY{p}{[}\PY{l+m+mi}{0}\PY{p}{]}\PY{p}{)}
          
          \PY{n}{counts} \PY{o}{=} \PY{n}{counts}\PY{o}{.}\PY{n}{drop}\PY{p}{(}\PY{l+s+s2}{\PYZdq{}}\PY{l+s+s2}{A65}\PY{l+s+s2}{\PYZdq{}}\PY{p}{)}
          \PY{n}{sns}\PY{o}{.}\PY{n}{barplot}\PY{p}{(}\PY{n}{counts}\PY{o}{.}\PY{n}{index}\PY{p}{,}  \PY{n}{counts}\PY{o}{.}\PY{n}{values}\PY{p}{,} \PY{n}{alpha}\PY{o}{=}\PY{o}{.}\PY{l+m+mi}{95}\PY{p}{,} \PY{n}{palette}\PY{o}{=}\PY{l+s+s2}{\PYZdq{}}\PY{l+s+s2}{Set3}\PY{l+s+s2}{\PYZdq{}}\PY{p}{,} \PY{n}{ax}\PY{o}{=}\PY{n}{ax}\PY{p}{[}\PY{l+m+mi}{1}\PY{p}{]}\PY{p}{)}
\end{Verbatim}


\begin{Verbatim}[commandchars=\\\{\}]
{\color{outcolor}Out[{\color{outcolor}138}]:} <matplotlib.axes.\_subplots.AxesSubplot at 0x23ee7470>
\end{Verbatim}
            
    \begin{center}
    \adjustimage{max size={0.9\linewidth}{0.9\paperheight}}{output_18_1.png}
    \end{center}
    { \hspace*{\fill} \\}
    
    Again, the left plot shows all the values appearing in the neighbourhood
for attribute savings: A61 9 time, all others, A63, A62 and A65 1 time.
We remove the original value, A65 and choose from three remaining values
(plot on the right). The value A61 has way more chances to be chosen
than the remaining two and the algorithm indeed chooses it as a mark.

    The nature of choosing the most frequent values in the neighbourhood
leads us to the assumption that by marking categorical values, the
distribution might converge to most frequent values in the dataset.
However, looking from the other perspective, the most frequent values
are also most likely to be chosen for marking, and therefore forced to
change to another, less frequent value (remember that the change of a
values is always forced if possible).

We can see how the distribution of values in categorical attributes
changes by applying fingerprinting on the data.


    % Add a bibliography block to the postdoc
    
    
    
    \end{document}
